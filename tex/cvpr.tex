% This version of CVPR template is provided by Ming-Ming Cheng.
% Please leave an issue if you found a bug:
% https://github.com/MCG-NKU/CVPR_Template.

\documentclass[review]{cvpr}
%\documentclass[final]{cvpr}

\usepackage{times}
\usepackage{epsfig}
\usepackage{graphicx}
\usepackage{amsmath}
\usepackage{amssymb}
\usepackage{caption}
\usepackage{subcaption}
\usepackage{gensymb}
\usepackage{stackrel}
\usepackage{float}
\newcommand{\X}{\mathbf{X}}
\newcommand{\R}{\mathbf{R}}
\newcommand{\D}{\mathcal{D}}
\newcommand{\dtmax}{\Delta \theta_\text{max}}

% Include other packages here, before hyperref.

% If you comment hyperref and then uncomment it, you should delete
% egpaper.aux before re-running latex.  (Or just hit 'q' on the first latex
% run, let it finish, and you should be clear).
\usepackage[pagebackref=true,breaklinks=true,colorlinks,bookmarks=false]{hyperref}

\def\cvprPaperID{****} % *** Enter the CVPR Paper ID here
\def\confYear{CVPR 2021}
%\setcounter{page}{4321} % For final version only

\begin{document}

%%%%%%%%% TITLE
\title{Exploiting and Refining Probabilistic Depth Estimates with RGB and Triangulating Light Curtain Fusion}

\author{Yaadhav Raaj ~ Siddharth Ancha ~ Joe Bartels ~ Robert Tamburo ~ Srinivasa G. Narasimhan\\
Carnegie Mellon University\\
{\tt\small \{ryaadhav, sancha, bartels, rtamburo, srinivas\}@cs.cmu.edu}
% For a paper whose authors are all at the same institution,
% omit the following lines up until the closing ``}''.
% Additional authors and addresses can be added with ``\and'',
% just like the second author.
% To save space, use either the email address or home page, not both
}

\maketitle

%%%%%%%%% ABSTRACT
\begin{abstract}
Programmable Triangulating Light Curtains are a low-cost, high speed and high resolution approach to depth sensing, but require the user to specify a 2.5D ruled surface over where sensing should occur. Objects that intersect this surface show up with a high intensity, while those that don't do not. We have devised a novel algorithm, that exploits prior depth distributions from either Monocular or Stereo RGB cameras to sample depth in regions of the world most uncertain using the Light Curtain. We then incorporate this data to get a refined depth estimate over time. We evaluate this setup with real world real-time experiments such as in the context of Advanced driver-assistance systems (ADAS)
\end{abstract}

\iffalse
seperately put the lidar pointcloud
monocular depth image
then lidar
then light curtain
teaser

Figure 2 from the Agile Depth Sensing   

mu + sigma - sigma * N

Fig 13 - Spread 1st 2nd and 3rd std

Fig 17 - Integers Numerical

Experiments - Put graph of Train from scratch in (its running)
Yes but this is because of the 0.7 thing?

Train defailt_expy_lc_ilim with prob mode 0.7
Experinment for Stereo Version
Experiment for Train with prob mode 1.0?
Experiment for Reducing the Itertaions or using diff policy

\fi


%%%%%%%%% BODY TEXT
\section{Introduction}

Programmable Triangulating Light Curtains were introduced last year, and have demonstrated viability in being a low cost (1k\$ vs lidar's ~25k\$), high spatial angular resolution (0.02$\degree$ vs lidar's  0.4$\degree$), and high frame-rate (~60fps vs lidar's ~20fps) sensor. They have demonstrated having strong returns in the presence of artifacts such as smoke, and also in seeing small targets such as pedestrians further away. They do however, require a user to specify at what depth per pixel one want's sensed, through providing a physically possible 2.5D ruled surface that intersects the world to provide returns. 

Conversely, depth estimation through RGB cameras have been a heavily researched area, but do not provide the realiability in estimates as compared to a Lidar, which can be cost prohibitive. This has made both RGB and Lidar's en-masse adoption in safety critical systems such as Advanced driver-assistance systems (ADAS) rare. 

Our work, looks into marrying both modalities to counter each sensor's limitations, through the use of Geometric, Probabilistic and Physical constraints. We begin by formulating an iterative Bayesian inference approach to depth sensing using the Light Curtain alone, leveraging on Depth Probability Volume (DPV) and corresponding 2D Uncertainty Field (UF) representations. We then build upon existing planning framework based on Dynamic Programming using the Light Curtain's Galvanometer velocity and acceleration constraints. Lastly, we expand upon Generating and Exploiting Probabilistic Depth Estimates using Deep Learning based approaches, and use that as a prior to drive Light Curtain placement for further refinement.

\section{Prior Work}

\textbf{Depth from Active Sensors: } Active sensors use a static or fixed scan light source / receiver to perceive depth. Long range outdoor depth from these such as commercially available Time-Of-Flight cameras \cite{quanergy} \cite{luminar} or LIDARs \cite{velodyne} \cite{ouster} provide dense metric depth with confidence values with wide usage in research \cite{Geiger2013IJRR} \cite{caesar2020nuscenes}  \cite{chang2019argoverse}. However, apart from low resolution, these sensors are difficult to procure and expensive, making everyday personal vehicle adoption challenging.

\textbf{Depth from Adaptive Sensors: } Adaptive sensors use a dynamically controllable light source / receiver instead. These have been making headway in the Long Range Outdoor space. Adaptive Sensing via focal length/baseline variation through the use of servos/motors \cite{Mohamed2018ActiveSP} \cite{4587671} \cite{Nakabo2005VariableBS} \cite{Schneider2018VisuallyGV}, directionally controlled Time-of-Flight Ranging using a MEMS mirror / laser \cite{dtof} \cite{9105183} \cite{pittaluga2020} \cite{8369664}, Gated Depth Imaging \cite{walz2020uncertainty} \cite{gruber2019gated2depth} \cite{10.1117/12.2078169} and finally, sampling specific depth profiles using Triangulation Light Curtains \cite{bartels2019Agile} \cite{wang2018programmable} \cite{Ancha_2020_ECCV} are just some examples. However, these methods do not seem to exploit or fuse data from RGB modalities yet.

\textbf{Depth from RGB: } Depth from Monocular, Stereo or Multi-Camera RGB has been extensively studied and deployed. We focus on a class of Probabilistic Depth estimation approaches that have reformulated the problem as a prediction of per-pixel depth distribution \cite{liu2019neural} \cite{yang2019inferring} \cite{chang2018pyramid} \cite{yao2018mvsnet} \cite{ilg2018uncertainty} \cite{xia2019generating}. Some of this work has actually passively exploited and refined \cite{liu2019neural} \cite{xia2019generating} the uncertainty in the depth values via Moving Cameras and Multi-View-Camera constraints, but have not used the capabilities of the slew of Adaptive Sensors available. 

We hope to fill this gap by investigating if a Probabilistic Depth representation from RGB sensors can be exploited by an Adaptive Sensor such as a Light Curtain to potentially match the precision of LIDARs but in a low cost manner..

\section{Sensor Setup}

\begin{figure}[h]
   \centering
   \begin{minipage}{0.5\textwidth}
       \centering
       \includegraphics[width=1.0\textwidth]{figures/LC.pdf} % first figure itself
   \end{minipage}\hfill
   % \begin{minipage}{0.3\textwidth}
   %     \centering
   %     \includegraphics[width=0.9\textwidth]{light_curtain_iso.png} % second figure itself
   % \end{minipage}
   \centering
   \caption{ \textbf{Below:}  Our Adaptive Sensor of choice, the Triangulation Light Curtain Device (LC) \cite{bartels2019Agile}, consisting of a laser, galvomirror and NIR camera. \textbf{Above:} The light curtain senses a ruled 3D surface extruding from a given top-down 2D curve we call a curtain. Surfaces that fall within the \textit{thickness} of the curtain, result in higher intensity shown in the NIR image.}
   \label{fig:lcdevice}  
\end{figure}

\begin{figure}[h]
    \centering
    \begin{minipage}{0.5\textwidth}
        \centering
        \includegraphics[width=1.0\textwidth]{figures/planesweep.png} % first figure itself
    \end{minipage}\hfill
    % \begin{minipage}{0.3\textwidth}
    %     \centering
    %     \includegraphics[width=0.9\textwidth]{light_curtain_iso.png} % second figure itself
    % \end{minipage}
    \centering
    \caption{Here, we sweep a planar 3D ruled surface or curtain across at various depths. As the curtain surface approaches the true surface, the intensity value slowly increases on the NIR image, due to the sensing location and curtain thickness. One can see how we could adaptively sense a scene to converge on the real depth with our device.}
    \label{fig:planesweep} 
 \end{figure}

The Light Curtain device consists of a rolling shutter NIR (Near-Infrared) camera rotated 90$\degree$ (that images planes in the world per pixel column), a Line Laser module and a Galvomirror (that generates planes of light in the world depending on the angle). The exact sensing location is obtained by intersecting (triangulating) these two planes, and sweeping this laser line creates a 3D ruled surface called a curtain. We can place a curtain along any surface by controlling the galvo and rolling shutter speed subject to it's constraints, making it adaptive in nature. Do note that the image and laser planes have some divergence, so their intersection results in a volume in space (bounded by purple points in Fig~\ref{fig:lcdevice}) with some \textit{thickness}, where any objects that intersect it result in higher intensities in the NIR image. This means that as the sensing location approaches the true surface, pixel intensities on NIR image increases (Fig.~\ref{fig:planesweep}). 

\begin{figure}[h]
   \centering
   \begin{minipage}{0.4\textwidth}
       \centering
       \includegraphics[width=1.0\textwidth]{figures/array.pdf}
   \end{minipage}\hfill
   \centering
   \caption{Setup for real-world experiments: FLIR Stereo camera pair, Light Curtain (LC) and OS2-128 Lidar used for ground truth validation.}
   \label{fig:sensorarray} 
\end{figure}

Real-world experiments are conducted using our array of sensors consisting of an RGB Stereo Camera Pair, Light Curtain device, and a 128 Beam Lidar for accuracy validation and later RGB depth estimation network training (Fig.~\ref{fig:sensorarray}). Simulated experiments are also conducted with KITTI dataset \cite{Geiger2013IJRR}, through a Light Curtain Simulator that uses the ground truth depthmap along with control in NIR instrinsics, Laser extrinsincs, Galvomirror speed, Laser Divergence/Thickness and Angle. 

% \begin{figure}[h]
%    \centering
%    \begin{minipage}{0.5\textwidth}
%        \centering
%        \includegraphics[width=1.0\textwidth]{figures/sim.png}
%    \end{minipage}\hfill
%    \centering
%    \caption{Light Curtain Simulator}
%    \label{fig:lcsimkitti} 
% \end{figure}
%\vspace{-.1in}




\section{Formulation}

We begin by looking at a Light Curtain only problem of adaptively discovering the depth of a scene. We wouldn't know the path along the rays/pixels on which an object may lie/intersect, hence we choose to go with a sampling based approach. We treat each ray as initially having either a uniform distribution, or a gaussian with a large variance in the center, and we attempt to formulate our problem with a Recursive Bayesian update approach.

\subsection{Representation}

\smallskip
Our state space is represented as a tensor of some fixed resolution [320x240] with each pixel in image $I$ encoding the depth value $\mathbf{d}(u,v)$ as a bernoulli distribution $P(\mathbf{d}(u,v))$. $D_{c}$ represents the list of depths quantizing the space of each pixel defined within $(d_{min}, d_{max})$ of some fixed size $N$ [64], resulting in a Depth Probability Volume (DPV) tensor of size [64, 240, 320]
\begin{align}
   D_{c}=\left\{ d_{0}...d_{n}\right\} \quad d_{i}=d_{min}+(d_{max}-d_{min})*t 
   \label{eq:d_candi}
   \\
   P(\mathbf{d}(u,v))=I(u,v)
   \nonumber\\
   \sum_{d}\left(P(\mathbf{d}(u,v))\right)=1\qquad \mathbb{E}[P(\mathbf{d}(u,v))]=\mathbf{d}(u,v)
   \label{eq:depth_dist}
   %\vspace{-.1in}
\end{align}

While an ideal sensor could choose plan a path to sample in the full 3D volume, our Light Curtain device only has control over a collapsed XZ  space. Hence, we select a subset of rays that correspond to a plane that we wish to sample, and generate an Uncertainty Field (UF) as follows:
\begin{align}
   P(\mathbf{d}(u))=\frac{\sum_{u,v}P(\mathbf{d}{(u,v)}).\boldsymbol{1}}{\sum_{u,v}\boldsymbol{1}}
   \nonumber \\
   \;where\;h_{min}>\mathbb{E}[P(\mathbf{d}(u,v))]>h_{max}
   \label{eq:collapse}
   %\vspace{-.1in}
\end{align}

\begin{figure}[h]
   \centering
   \begin{minipage}{0.5\textwidth}
       \centering
       \includegraphics[width=1.0\textwidth]{figures/bev.pdf}
   \end{minipage}\hfill
   \centering
   \caption{Our state space consisting of a Depth Probability Volume (DPV) its corrsp Bird's Eye Uncertainty Field (UF)}
\end{figure}

\subsection{Curtain Planning}

With the Uncertainty Field (UF) extracted from the state space, we can use this to figure out where to place light curtains. We build upon prior work solving Light Curtain placement as a Constraint Optimization and Dynamic Programming problem. A single light curtain placement is defined by a set of $T$ control points $\{\X_t\}_{t=1}^T$. We wish to maximize the objective $J(\X_1, \dots, \X_T) = \sum_{t=1}^T UF(\X_t)$ where $UF(\X)$ is the uncertaintiy field probabilies at the anchor location of $\X$.

The control points $\{\X_t\}_{t=1}^T$, where each $\X_t$ lies on the the camera ray $\R_t$, must be chosen to satisfy the physical constraints of the light curtain device: $|\theta(\X_{t+1}) - \theta(\X_t)| \leq \Delta \theta_\text{max}$ with $\theta_\text{max}$ being the maximum angular velocity of the Galvomirror. The problem is also discretized such that $\X_{t} \in D_c\ $ and also lies along $\R_t$
\begin{align}
    &\arg \max_{\{\X_t\}_{t=1}^T} \sum_{t=1}^T UF(\X_t) \qquad \text{where}\ \X_t \in D_c\ \nonumber\\
    &\text{subject to}\ |\theta(\X_{t+1}) - \theta(\X_t)| \leq \dtmax,\ \forall 1 \leq t < T
    \label{eq:constraint}
\end{align}

\begin{figure}[h]
   \centering
   \begin{minipage}{0.5\textwidth}
       \centering
       \includegraphics[width=1.0\textwidth]{figures/planner.pdf}
   \end{minipage}\hfill
   \centering
   \caption{Left: Light Curtain constraint graph subject to max angular velocity of Galvomirror. Right: Placing an optimal curtain along the highest probability region per column of rays}
\end{figure}

In the figure above, we have placed a single curtain along the highest probability region per column of rays, but ideally, we should place more curtains spanning the uncertaintiy of those distributions. To do this, we generate corresponding entropy fields $H(\X)_{i}$ to be fed to the planner from $UF(\X)$ based on two approaches: $m0$ attempts to normalize each ray's distribution $P(\mathbf{d}(u))$ and warp the distribution such that a selected span from the mean is maximized. $m1$ attempts to sample a point on the ray given $P(\mathbf{d}(u))$. 

\begin{figure}[h]
   \centering
   \begin{minipage}{0.5\textwidth}
       \centering
       \includegraphics[width=1.0\textwidth]{figures/fields.pdf}
   \end{minipage}\hfill
   \centering
   \caption{We look at a depth distribution of one of the rays in UF (yellow line), and figure out where additional curtains (blue points) can be placed such as to maximize information gained. Observe that $m1$ is able to handle multimodal distributions}
   \label{fig:m0m1}
\end{figure}

As seen in Fig. ~\ref{fig:m0m1}, strategy $m0$ is able to generate fields that adaptively place additional curtains around a consistent span around the mean, but is unable to do so in cases of multimodal distributions. $m1$ on the other hand is able to place a curtain around the 2nd modality, albeit with a lower probability. The inconsistency from ray to ray in $m1$ however, may be impossible to image due to it exceeding the acceleration bounds, hence a spline fit is used with control points every 5 to 10 rays, resulting in an imagable but non-flat curtain placement. We see the effects of both in later experiments.

\subsection{Observation Model}

We now have returns from curtains $\mathbf{C}$ with each $\mathbf{C}_{i}$ containing $[x,y,z,i]$, the 3D position and intensity value in the same spatial resolution as $I$, planned based on a particular policy, conditioned on our prior distribution $P(\mathbf{d}_{t})$ at time $t$. We need to convert $\mathbf{C}_{i}$ into a likelihood distribution $P(\mathbf{c}_{t}|\mathbf{d}_{t})$, such that this Hidden Markov Model representation and sum of log likelihoods holds true:
\small
\begin{align}
   P\left(\mathbf{d}_{0},...,\mathbf{d}_{T},\mathbf{c}_{1},...,\mathbf{c}_{T}\right)= 
   P\left(\mathbf{d}_{0}\right)\mathbf{\mathbf{\prod}_{\mathrm{t=1}}^{\mathrm{T}}}P\left(\mathbf{c}_{t}|\mathbf{d}_{t}\right)P\left(\mathbf{d}_{t}|\mathbf{d}_{t-1}\right) \nonumber\\
   \log\left(P\left(\mathbf{c}_{t}|\mathbf{d}_{t}\right)\right)=\stackrel[i=0]{n}{\sum}\left(\log\left(P\left(\mathbf{c_{\mathit{i}}}_{t}|\mathbf{d}_{t}\right)\right)\right)
   \label{eq:hmm}
\end{align}
\normalsize

We represent $P\left(\mathbf{c_{\mathit{i}}}_{t}|\mathbf{d}_{t}\right)$ as a linear combination of a gaussian and a uniform distribution, where $\sigma$ is a function of the thickness of the light curtain as described earlier where $t\in[-1..1]$. $t$ is a function of the intensity value $i$, based on two possible profiles where $z$ either takes a value of 0.5 or 1.0, and $m$ is some control factor we tune:

\small
\begin{align}
   P\left(\mathbf{c_{\mathit{i}}}_{t}|\mathbf{d}_{t}\right)=\frac{\mathcal{N}\left(D_{c},\mu_{c},\sigma_{c}\right)\left(t\right)+U\left(D_{c}\right)(1-t)}{\sum\left(\mathcal{N}\left(D_{c},\mu_{c},\sigma_{c}\right)\left(t\right)+U\left(D_{c}\right)(1-t)\right)} \\
   t=\left(\frac{-1}{\left(z\right)+\left(mi\right)}\right)+1
   \label{eq:dist}
\end{align}
\normalsize

\begin{figure}[h]
   \centering
   \begin{minipage}{0.5\textwidth}
       \centering
       \includegraphics[width=1.0\textwidth]{figures/graphs.png}
   \end{minipage}\hfill
   \centering
   \caption{The effect on the posterior when getting a close to 0 intensity return from the Light Curtain when $z$ is either 1.0 or 0.5}
   \label{fig:updatemodel}
\end{figure}

As seen in Fig. ~\ref{fig:updatemodel}, as the intensity of the light curtain varies from 0 to 1 on the $x$ axis, we vary the value of $t$ on the $y$ axis. When $z$ is 0.5, low intensity returns (eg. $<$ 0.1) results in a negative $t$ value. This means that when the intensity is high, a higher return results in a larger peak probability for both cases when $z$ is 1.0 or 0.5. But when the intensity is low or close to 0, $z=1.0$ causes $P\left(\mathbf{c}_{t}|\mathbf{d}_{t}\right)$ to tend to a uniform distribution, but $z=0.5$ to tend to an inverted gaussian. With this case, the rationale is that having no return at a location doesn't mean that we have derived no information at all, but rather, we know that this particular location is less likely to have an object and the rest of the the locations along the ray are equally uncertain/uniform. We show in experiments that this results in significantly faster convergence, and we call this the \textit{Inverting Gaussian model}

\section{Light Curtain only Experiments}

Can we estimate depth using just the Light Curtain? In this initial baseline, we attempt to track the Uncertainty Field (UF) depth error by computing the RMSE error metric $\sqrt{\stackrel[i=1]{n}{\sum}\frac{\left(\mathbb{E}\left(UF\left(u,q\right)\right)-\mathbf{d_{gt}}(u)_{i}\right)^{2}}{n}}$ against the ground truth. We use 3 scenarios: One in a KITTI driving scene using the LC simulator (a), one indoors in the basement using the both a simulated and real Light Curtain (b), and lastly in various outdoor driving scenes with the real device (c0/c1/c2). 

\begin{figure}[h]
   \centering
   \begin{minipage}{0.5\textwidth}
       \centering
       \includegraphics[width=1.0\textwidth]{figures/exp.png}
   \end{minipage}\hfill
   \centering
   \caption{Scenarios (a), (b), (c) from left to right}
   \label{fig:exp}
\end{figure}

\textbf{Planar Sweep:} A simple sanity check between simulation and the real sensor involves involves performing a uniform sweep across the scene in (b). As seen in Fig.~\ref{fig:planarsweep}, our simulated Light Curtain (LC) is able to reasonably match the real device. We also demonstrate how decreasing the steps the curtain takes reduces runtime but increases RMSE. Results seen in Tab.~\ref{table:t1}
 
\begin{figure}
   \centering
   \begin{minipage}{0.4\textwidth}
       \centering
       \includegraphics[width=1.0\textwidth]{figures/sweep.png}
   \end{minipage}\hfill
   \centering
   \caption{We demonstrate corroboration between simulated and real Light Curtain by sweeping several planes across the scene (b). Colored pointcloud is the estimated depth, and Lidar ground truth in yellow. \textbf{Left:} LC simulated from the Lidar Depth. \textbf{Right:} Using the real LC}
   \label{fig:planarsweep}
\end{figure}

%\vspace{-.1in}
\begin{table}[h]
   \centering
   \resizebox{0.7\linewidth}{!}{
   \begin{tabular}{|l|l|l|}
   \hline
    Policy&  Runtime/s&  RMSE/m\\ \hline
    Sweep 50 $\mathbf{C}$ Step 0.25m (Sim) &-  &1.156  \\ \hline
    Sweep 25 $\mathbf{C}$ Step 0.5m  (Sim) &-  &1.374  \\ \hline
    Sweep 50 $\mathbf{C}$ Step 0.25m (Real) &2  &1.284  \\ \hline
    Sweep 25 $\mathbf{C}$ Step 0.5m  (Real) &1  &1.574  \\ \hline
    Sweep 12 $\mathbf{C}$ Step 1.0m (Real) &0.5  &1.927  \\ \hline
   \end{tabular}}

   \caption{We show that sampling the scene by placing more curtains results in better depth accuracy}
   \label{table:t1}
   %\vspace{.1in}
   
%    \resizebox{0.7\linewidth}{!}{
%     \begin{tabular}{|l|l|l|}
%     \hline
%      Policy&  Runtime/s&  RMSE/m\\ \hline
%      Sweep $\mathbf{C}$ Step 0.25m (Dyn) &2  &1.276  \\ \hline
%      Sweep $\mathbf{C}$ Step 0.5m  (Dyn) &1  &1.532  \\ \hline
%      Sweep $\mathbf{C}$ Step 1.0m (Dyn) &0.5  &2.013  \\ \hline
%      Sweep $\mathbf{C}$ Step 0.25m (Fixed) &2  &1.218  \\ \hline
%      Sweep $\mathbf{C}$ Step 0.5m  (Fixed) &1  &1.658  \\ \hline
%      Sweep $\mathbf{C}$ Step 1.0m (Fixed) &0.5  &2.290  \\ \hline
%     \end{tabular}} 
   
%    \caption{$\sigma_{c}$ in our model $P\left(\mathbf{d}_{u,v}^{c_{k}}\right)$ being dynamic as a function of curtain thickness as opposed to being fixed, results in better depth estimates}
%    \label{table:t2}
\end{table}

% \textbf{Effect of Dynamic Sigma:} Earlier, we had noted how $\sigma(u,v,d_{u,v}^{c})$ defined for each $c_{k}$ measurement in $P\left(\mathbf{d}_{u,v}^{c_{k}}\right)$ is a function of the thickness of the curtain. We also experiment by making $\sigma(u,v,d_{u,v}^{c})$ fixed. We observe that it being a function of the curtain thickness is critical to better performance over larger steps/placements. Results in Tab.~\ref{table:t2}

% %\vspace{-.1in}
% \noindent
% \begin{table}[h]
%    \centering
%    \resizebox{0.7\linewidth}{!}{
%    \begin{tabular}{|l|l|l|}
%    \hline
%     Policy&  Runtime/s&  RMSE/m\\ \hline
%     Sweep $\mathbf{C}$ Step 0.25m (Dyn) &2  &1.276  \\ \hline
%     Sweep $\mathbf{C}$ Step 0.5m  (Dyn) &1  &1.532  \\ \hline
%     Sweep $\mathbf{C}$ Step 1.0m (Dyn) &0.5  &2.013  \\ \hline
%     Sweep $\mathbf{C}$ Step 0.25m (Fixed) &2  &1.218  \\ \hline
%     Sweep $\mathbf{C}$ Step 0.5m  (Fixed) &1  &1.658  \\ \hline
%     Sweep $\mathbf{C}$ Step 1.0m (Fixed) &0.5  &2.290  \\ \hline
%    \end{tabular}}
%    \caption{$\sigma_{c}$ in our model $P\left(\mathbf{d}_{u,v}^{c_{k}}\right)$ being dynamic as a function of curtain thickness as opposed to being fixed, results in better depth estimates}
%    \label{table:t2}
% \end{table}

\textbf{Effect of Inverting Gaussian Model:} Our Observation Model ensures that the sensor distribution tends to an Inverted Gaussian when intensities are low, instead of a Uniform distribution. We see significantly improved performance when low intensities tend to an Inverted Gaussian instead. Results seen in Fig.~\ref{fig:figure01} ~\ref{fig:invgau}

\begin{figure}[h]
    \centering
    \begin{minipage}{0.5\textwidth}
        \centering
        \includegraphics[width=0.49\textwidth]{figures/Figure_0.pdf}
        \includegraphics[width=0.49\textwidth]{figures/Figure_1.pdf}
    \end{minipage}\hfill
    \centering
    \caption{We get better depth estimate over increased sampling iterations, when low intensity values sampled by the light curtain generate an Inverted Gaussian Depth Distribution, as opposed to a uniform one in both simulation and real world experiments}
    \label{fig:figure01}
\end{figure}

 \begin{figure}[h]
    \centering
    \begin{minipage}{0.5\textwidth}
        \centering
        \includegraphics[width=0.95\textwidth]{figures/combined2.png}
    \end{minipage}\hfill
    \centering
    \caption{Looking at the top-down Uncertainty Field (UF), we see per pixel distributions in Blue and the GT in Red. We start with a gaussian prior with a large $\sigma$, take measurements and apply the bayesian update. \textbf{Left:} Policies $\pi_{0}$ and $\pi_{1}$ where low intensities result in no information (Uniform Distribution). \textbf{Right:} Where low intensities result in an Inverted Gaussian based on our Sensor Model}
    \label{fig:invgau} 
\end{figure}

 \textbf{Effect of Placing more Curtains:} In both policies, placing more light curtains results in much faster convergence, at the cost of increased runtime. Results seen in Fig.~\ref{fig:figure34}

\begin{figure}[h]
    \centering
    \begin{minipage}{0.5\textwidth}
        \centering
        \includegraphics[width=0.49\textwidth]{figures/Figure_3.pdf}
        \includegraphics[width=0.49\textwidth]{figures/Figure_4.pdf}
    \end{minipage}\hfill
    \centering
    \caption{Placing more curtains results in faster convergence to the true depth at the cost of runtime per iteration}
    \label{fig:figure34}
\end{figure}

% While this is a reasonable approach to estimating depth, it is also slow, as it requires placing many curtains. 


\begin{figure*}[t!]
   \includegraphics[width=1.0\textwidth]{figures/network.pdf}
   \caption{Our network takes in RGB images to generate a Depth Probability Volume (DPV). We then drive our Triangulation Light Curtain's laser to plan and place curtains on regions that are uncertain and refine it. This is then fed back on the next timestep to get much more refined DPV estimate.}
   \label{fig:network}
\end{figure*}
   
\section{Depth from Light Curtain + RGB Fusion}

\subsection{Motivation}

While starting from a uniform or gaussian prior with a large uncertainity is a valid option, it is slow to converge. Furthermore, a Light Curtain's only means of depth estimation is extracted primarily along the ruled placement of the curtain, at least based on our above placement policies. We would ideally like to use information from a Monocular RGB camera or Stereo Pair to initialize our prior, with a similar DPV representation. To this end, a Deep Learning based architecture is ideal, and we also reason that such an architecture could potentially learn to fuse/incorporate information from both modalities better.

\subsection{Structure of Network}

The first step is to build a network (Fig.~\ref{fig:network}) that can generate DPV's similar to our light curtain only estimation strategy from RGB images. To this end, we build upon the MVSNet/PSMNet \cite{chang2018pyramid} \cite{yao2018mvsnet} architecture. N images, usually 2 $I_{0}, I_{1}$ are fed into encoders that share weights, and the features are then warped into different fronto-parallel planes of the reference image $I_{0}$ using pre-computed $R_{i}^{0}, t_{i}^{0}$. Further convolutions are run to generate a low resolution DPV $dpv_{\mathrm{t}}^{\mathrm{l0}}$ [80, 96] where the log softmax operator is applied and regressed on. The transform between the camera's act as a constraint, forcing the feature maps to respect depth to channel correspondence. The add operator into a common feature map is similar to adding probabilities in log-space. 

This is then fed into the DPV Fusion Network (Set of 3D Convolutions) that incorporate a downsampled version of $dpv_{\mathrm{t-1}}^{\mathrm{L}}$ along with the the light curtain DPV that we had applied recursive bayesian updates on $dpv_{\mathrm{t-1}}^{\mathrm{lc}}$, and a residual is computed and added back to $dpv_{\mathrm{t}}^{\mathrm{l0}}$ to generate $dpv_{\mathrm{t}}^{\mathrm{l1}}$ to be regressed upon similarly. Occasionally, we train without this feedback by inputting a uniform distribution. Finally, this is then passed into a decoder with skip connections to generate a high resolution DPV $dpv_{\mathrm{t}}^{\mathrm{L}}$. This is then used to plan and place light curtains, from which we generate a new $dpv_{\mathrm{t}}^{\mathrm{lc}}$ to be fed in the next stage.

\subsection{Loss Functions}

\textbf{Soft Cross Entropy Loss:} We build upon the ideas found in \cite{Yang-2019-118007} and use a soft cross entropy loss function, with the ground truth lidar depthmap becoming a gaussian DPV with $\sigma_{gt}$ instead of a one hot vector. This way, when taking $\mathbb{E}\left(dpv^{gt}\right)$ we get the exact depth value instead of an approxmiation limited by the depth quantization $\D$. We also make the quantization slightly non-linear to have more steps between objects that are closer to the camera.
\small
\begin{align}
   l_{sce}=\frac{-\sum_{i}\sum_{d}\left(dpv^{\mathrm{\{l0,l1,L\}}}*log\left(dpv^{gt}\right)\right)}{n} \\
   \D=\{d_{0},\dots,d_{N-1}\};d_{q}=d_{\text{min}}+(d_{\text{max}}-d_{\text{min}})\cdot q^{pow}
  \label{eq:seloss}
\end{align}
\normalsize

\textbf{L/R Consistency Loss:} We train on both the Left and Right Images of the stereo pair where the Projection matrices $P_{l}, P_{r}$ are known \cite{godard2017unsupervised}. We enforce predicted Depth and RGB consistency by warping the Left Depthmap/Projected RGB into the Right Camera and vice-versa, and minimize the following metric:
\small
\begin{align}
   D_{l}=\mathbb{E}\left(dpv_{l}^{L}\right)\qquad D_{r}=\mathbb{E}\left(dpv_{r}^{L}\right) \\
   l_{dcl}=\frac{1}{n}\sum_{i}\left(\frac{\left|D_{\{l,r\}}-w\left(D_{\{r,l\}},P_{\{l,r\}}\right)\right|}{D_{\{l,r\}}+w\left(D_{\{r,l\}},P_{\{l,r\}}\right)}\right) \\
   l_{rcl}=\frac{1}{n}\sum_{i}\left(||I_{\{l,r\}}-w\left(I_{\{r,l\}},D_{\{l,r\}},P_{\{l,r\}}\right)||_{1}\right)
  \label{eq:lrcons} 
\end{align}
\normalsize

\textbf{Edge aware Smoothness Loss:} We ensure that neighbouring pixels have consistent surface normals, except on the edges/boundaries of objects with the Sobel operator $S_{x}, S_{y}$ via the term:
\small
\begin{align}
   l_{s}=\frac{1}{n}\sum_{i}\left(\left|\frac{\partial I}{\partial x}\right|e^{-|S_{x}I|}+\left|\frac{\partial I}{\partial y}\right|e^{-|S_{y}I|}\right)
  \label{eq:smooth} 
\end{align}
\normalsize

\subsection{Datasets}

We train and validate our algorithms on the KITTI dataset. We then trained the same network by initializing on those weights, but using our custom dataset to evaluate our algorithms on our sensor platform / jeep.

\begin{figure*}[h!]
  \includegraphics[width=1.0\textwidth]{figures/p1.png}
  \includegraphics[width=1.0\textwidth]{figures/p2.png}
  \caption{In KITTI + Simulated Light Curtain, we note improved depthmaps when Monocular inputs are fused with Light Curtain inputs. Also, we note that the same architecture can be used Stereo Depth and Lidar Upsampling}  
  \label{fig:images2} 
\end{figure*}

\begin{figure*}[h!]
  \includegraphics[width=1.0\textwidth]{figures/p4.png}
  \includegraphics[width=1.0\textwidth]{figures/p5.png}
  \caption{In Real World Experiments, we are able to see the monocular scale ambiguity in domain specific scenarios get corrected by the Light Curtain, and we are able to see correction in any arbitary scene provided to the system as well}  
  \label{fig:images3} 
\end{figure*}

\begin{figure*}[h!]
  \includegraphics[width=1.0\textwidth]{figures/p6.png}
  \caption{We show the internal state of the bayesian update at Iteration 0 and Iteration 5. Starting with a prior DPV from Monocular Depth estimation, we show the convergence of the sensor's laser and curtain profile on an object 15m away}  
  \label{fig:images4} 
\end{figure*}

\begin{figure*}[h!]
   \includegraphics[width=1.0\textwidth]{figures/lastone4.png}
   \caption{Monocular RGB alone suffers from scale ambiguity but does give an inital uncertain depth estimate on a car 15m away. Iterating on Light Curtain measurements from a mean-centered gaussian prior alone gives a more accurate depth but with a noisy profile, but starting with the RGB DPV results in a more accurate and smoother profile.}  
   \label{fig:images5} 
 \end{figure*}

% \begin{figure}[h]
%    \includegraphics[width=0.46\textwidth]{figures/lastone.png}
%    \caption{We show the internal state of the bayesian update at Iteration 0 and Iteration 5}  
%    \label{fig:images5} 
%  \end{figure}

\section{Light Curtain + RGB Fusion Experiments} 

Here we consider the RMSE metric against the entire depthmap as opposed to just the Uncertainity Field (UF) as $\sqrt{\stackrel[i=1]{n}{\sum}\frac{\left(\mathbb{E}\left(\mathbf{d}_{u,v}\right)-\mathbf{d_{gt}}(u,v)_{i}\right)^{2}}{n}}$ against our ground truth.

\textbf{Effects of our loss function:} We do some simple experiments, training the task of monocular depth estimation, and we explore the effects of enabling various loss functions:
\noindent
\begin{table}[h]
  \centering
  \resizebox{0.5\linewidth}{!}{
  \begin{tabular}{|l|l|}
  \hline
   Parameters&  RMSE/m\\ \hline
   $\sigma_{gt}=0.05$ &3.24  \\ \hline
   $\sigma_{gt}=0.2$ &3.16  \\ \hline
   $\sigma_{gt}=0.3$ &3.06  \\ \hline
   $\sigma_{gt}=0.3$ with $l_{dcl}, l_{rcl}$  &2.93  \\ \hline
   $\sigma_{gt}=0.3$ with $l_{dcl}, l_{rcl}, l_{s}$  &2.90  \\ \hline
  \end{tabular}}
  \caption{Effects of Soft Cross Entropy ($\sigma_{gt}$), Left/Right Consistency ($l_{dcl}, l_{rcl}$), Smoothness losses ($l_{s}$) on Monocular Depth Estimation}
  \label{table:xx}
\end{table}

\vspace{-.1in}
We note succesively improving performance as we increase $\sigma_{gt}$, with poorer performance when the depth is effectively encoded as a one-hot vector (eg. $\sigma_{gt}=0.05$), since the depth was more likely to be forced into one of the categories in $\D$. Adding in $l_{dcl}, l_{rcl}$ and $l_{s}$ improved performance further, but we did not see any major performance improvement in varying $q^{pow}$ with $\sigma_{gt}$ was increased.

\textbf{Effect of Stereo Inputs:} We experiment with passing in a Monocular Pair at $t, t-1$, or a Stereo Pair at $t$, with $R,t$ known in both cases. We note significantly better performance with Stereo input. Note that our method can generalize to any N camera setup (Fig.~\ref{fig:stereodpv}).

\textbf{Effect of DPV Fusion Network:} We consider the task of Lidar Upsampling. The Velodyne Lidar in the KITTI dataset, can be converted into a low resolution depthmap, and consequently a low res DPV we call $dpv_{t}^{gt}$ . We could then fuse both $dpv_{t}^{l0}$ and $dpv_{t}^{gt}$ to generate $dpv_{t}^{l1}$ using Bayesian Inference. Alternatively, we could feed both of those inputs into our DPV Fusion Network above. We note improved performance in this upsampling task when using this approach as seen in Fig.~\ref{fig:stereodpv}. 

\begin{figure}[H]
  \centering
  \begin{minipage}{0.5\textwidth}
      \centering
      \includegraphics[width=0.49\textwidth]{figures/Figure_6.pdf}
      \includegraphics[width=0.49\textwidth]{figures/Figure_7.pdf}
  \end{minipage}\hfill
  \centering
  \caption{ \textbf{Left:} Same network but having a Stereo Pair at $t$ passed in instead of Monocular Pair at $t, t-1$. \textbf{Right:} Fusing the GT Lidar data with $dpv_{t}^{l0}$ to generate $dpv_{t}^{l1}$ and $dpv_{t}^{L}$ with Bayesian Inference vs DPV Fusion Network}
  \label{fig:stereodpv} 
\end{figure}

\textbf{Effect of a Stronger Prior:} Here, we show that a prior DPV generated from a monocular RGB camera as opposed to a mean-centered gaussian with a large $\sigma$ yields higher accuracy and faster convergence (Fig.~\ref{fig:images5}, Fig.~\ref{fig:prior}).

\begin{figure}[h]
  \centering
  \begin{minipage}{0.5\textwidth}
      \centering
      \includegraphics[width=0.49\textwidth]{figures/figure_X.pdf}
  \end{minipage}\hfill
  \centering
  \caption{Starting from a Prior distribution from a Monocular Depth Network as opposed to a mean-centered gaussian with a large $\sigma$ leads to faster discovery and convergence of true depth}
  \label{fig:prior} 
\end{figure}

\textbf{Effect of a Light Curtain Fusion:} Here, we train regular Monocular Depth Estimation without Light Curtain Feedback, and one where we enable $dpv_{t}^{lc}$ to be planned and fed-back on the next stage, on the KITTI dataset with our Light Curtain Simulator. We observed quantitative Fig.~\ref{fig:images3} and qualitative Fig.~\ref{fig:lfusion} performance improvement of depth with Monocular input, and minor improvement with Stereo.

\begin{figure}[h]
  \centering
  \begin{minipage}{0.5\textwidth}
      \centering
      \includegraphics[width=0.49\textwidth]{figures/Figure_10.pdf}
      \includegraphics[width=0.49\textwidth]{figures/Figure_11.pdf}
  \end{minipage}\hfill
  \centering
  \caption{Monocular (Left) and Stereo (Right) Depth Estimation show improvement when we enabled feedback of the sensed Light Curtain DPV at epoch 16 when training on KITTI dataset with light curtain simulator}
  \label{fig:lfusion} 
\end{figure}

\subsection{Future Work}

We have demonstrated the first known work that has leveraged uncertainity in RGB cameras to drive an Adaptive Sensor such as a Light Curtain, in the context and large range operations of ADAS. Our approach \textit{can generalize} to any sensor that uses the principle of driving a laser or light source to specific pixels that are uncertain, and can benefit from depth uncertainity information of a pixel. 

Normally unincident and high reflectivity surfaces show poor intensity returns, so we hope to build a better sensor model that utilizes albedo and normal information. We could also extend this to temporaly inconsistent scenes (fast moving vehicle) by modelling scene flow.

\iffalse
Fig 15 maybe use the first image thing
Train Experiment on Fig 18 to show early LC feedback negative effect
Complete Future Work / Conclusion
\fi 






  

{\small
\bibliographystyle{ieee_fullname}
\bibliography{egbib}
}

\end{document}


\section{Light Curtain only Experiments}

In this initial baseline, we attempt to track the Uncertainty Field (UF) depth error by computing the RMSE error metric $\sqrt{\stackrel[i=1]{n}{\sum}\frac{\left(\mathbb{E}\left(P(\mathbf{d}(u)_{i})\right)-\mathbf{d_{gt}}(u)_{i}\right)^{2}}{n}}$ against the ground truth. We use 3 scenarios: One in a KITTI driving scene using the LC simulator (a), one indoors in the basement using the both a simulated and real Light Curtain (b), and lastly in various outdoor driving scenes with the real device (c0/c1/c2).

\begin{figure}[h]
   \centering
   \begin{minipage}{0.5\textwidth}
       \centering
       \includegraphics[width=1.0\textwidth]{figures/exp.png}
   \end{minipage}\hfill
   \centering
   \caption{Scenarios (a), (b), (c) from left to right}
   \label{fig:exp}
\end{figure}

\textbf{Planar Sweep:} A simple sanity check between simulation and the real sensor involves involves performing a uniform sweep across the scene in (b). 

\begin{figure}
   \centering
   \begin{minipage}{0.5\textwidth}
       \centering
       \includegraphics[width=1.0\textwidth]{figures/sweep.png}
   \end{minipage}\hfill
   \centering
   \caption{Doing a simple planar sweep across the scene. Colored pointcloud is the estimated depth, and Lidar ground truth in yellow. Left: LC simulated from the Lidar Depth. Right: Using the real LC}
   \label{fig:planarsweep}
\end{figure}

As seen in Fig. ~\ref{fig:planarsweep}, our simulated Light Curtain (LC) is able to reasonably match the real device. We also demonstrate how decreasing the steps the curtain takes reduces runtime but increases RMSE. Results seen in ~\ref{table:t1}

\noindent
\begin{table}[h]
   \centering
   \resizebox{0.9\linewidth}{!}{
   \begin{tabular}{|l|l|l|}
   \hline
    Policy&  Runtime/s&  RMSE/m\\ \hline
    Sweep 50 $\mathbf{C}$ Step 0.25m (Sim) &-  &1.156  \\ \hline
    Sweep 25 $\mathbf{C}$ Step 0.5m  (Sim) &-  &1.374  \\ \hline
    Sweep 50 $\mathbf{C}$ Step 0.25m (Real) &2  &1.284  \\ \hline
    Sweep 25 $\mathbf{C}$ Step 0.5m  (Real) &1  &1.574  \\ \hline
    Sweep 12 $\mathbf{C}$ Step 1.0m (Real) &0.5  &1.927  \\ \hline
   \end{tabular}}
   \caption{Error wrt to time and number of Sweep steps}
   \label{table:t1}
\end{table}

\textbf{Effect of Dynamic Sigma:} Earlier, we had noted how $\sigma_{c}$ defined for each $C_{i}$ measurement in $P\left(\mathbf{c}_{t}|\mathbf{d}_{t}\right)$ is a function of the thickness of the curtain. We also experiment by making $\sigma_{c}$ fixed. We observe that $\sigma_{c}$ being a function of the curtain thickness is critical to better performance over larger steps/placements. Results seen in ~\ref{table:t2}

\noindent
\begin{table}[h]
   \centering
   \resizebox{0.9\linewidth}{!}{
   \begin{tabular}{|l|l|l|}
   \hline
    Policy&  Runtime/s&  RMSE/m\\ \hline
    Sweep $\mathbf{C}$ Step 0.25m (Dyn) &2  &1.276  \\ \hline
    Sweep $\mathbf{C}$ Step 0.5m  (Dyn) &1  &1.532  \\ \hline
    Sweep $\mathbf{C}$ Step 1.0m (Dyn) &0.5  &2.013  \\ \hline
    Sweep $\mathbf{C}$ Step 0.25m (Fixed) &2  &1.218  \\ \hline
    Sweep $\mathbf{C}$ Step 0.5m  (Fixed) &1  &1.658  \\ \hline
    Sweep $\mathbf{C}$ Step 1.0m (Fixed) &0.5  &2.290  \\ \hline
   \end{tabular}}
   \caption{$\sigma_{c}$ in generated $P\left(\mathbf{c}_{t}|\mathbf{d}_{t}\right)$ being fixed vs being dynamic as a function of curtain thickness with real LC}
   \label{table:t2}
\end{table}

\textbf{Effect of Inverting Gaussian Model:} Our Observation Model ensures that the sensor distribution tends to an inverted gaussian when intensities are low. We test the effect on (a) and (b) in the above case, and in cases where low intensities tend to a uniform distribution. We see significantly improved performance when low intensities tend to an Inverted Gaussian rather than a uniform distribution. Results seen in ~\ref{fig:figure01} ~\ref{fig:invgau}

\begin{figure}[h]
    \centering
    \begin{minipage}{0.5\textwidth}
        \centering
        \includegraphics[width=0.49\textwidth]{figures/Figure_0.pdf}
        \includegraphics[width=0.49\textwidth]{figures/Figure_1.pdf}
    \end{minipage}\hfill
    \centering
    \caption{RMSE of Depth in UF over every iteration in scenarios left/(a) right/(b). Note the largely improved performance when low intensities tend to an Inverted Gaussian}
    \label{fig:figure01}
\end{figure}

 \begin{figure}[h]
    \centering
    \begin{minipage}{0.5\textwidth}
        \centering
        \includegraphics[width=0.49\textwidth]{figures/InvGau.png}
        \includegraphics[width=0.49\textwidth]{figures/Uniform.png}
    \end{minipage}\hfill
    \centering
    \caption{\textbf{Left:} Policies $m0$ and $m1$ where low intensities result in no information (Uniform Distribution). \textbf{Right:} Where low intensities result in an Inverted Gaussian based on our Sensor Model}
    \label{fig:invgau}
\end{figure}

 \textbf{Effect of Placing more Curtains:} In both policies, placing more light curtains results in much faster convergence, at the cost of increased runtime. Results seen in ~\ref{fig:figure34}

\begin{figure}[h]
    \centering
    \begin{minipage}{0.5\textwidth}
        \centering
        \includegraphics[width=0.49\textwidth]{figures/Figure_3.pdf}
        \includegraphics[width=0.49\textwidth]{figures/Figure_4.pdf}
    \end{minipage}\hfill
    \centering
    \caption{Placing more curtains results in faster convergence to the ground truth}
    \label{fig:figure34}
\end{figure}

% While this is a reasonable approach to estimating depth, it is also slow, as it requires placing many curtains. 


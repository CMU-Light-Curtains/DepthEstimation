\section{Prior Work}

Active Perception is a well studied problem of actively controlling sensors for improved perception \cite{bajcsy1988active} \cite{wilkes1994active} \cite{DBLP:journals/corr/BajcsyAT16} \cite{DBLP:journals/corr/abs-1811-08067}. These have been used in hand-in-eye robot setups for object detection and manipulation tasks \cite{sergey2017} \cite{7759446}, in Field Robots for Next-Best-View (NBV) Planning \cite{coslam2012} \cite{8098709} \cite{7989679}, and also in the domain of self-driving for depth sensing \cite{Ancha_2020_ECCV} \cite{9105252}.

Apart from Moving Cameras, Active Perception systems have been built by focal length/baseline variation through the use of servos/motors \cite{Mohamed2018ActiveSP} \cite{4587671} \cite{Nakabo2005VariableBS} \cite{Schneider2018VisuallyGV}, directionally controlled Time-of-Flight Ranging using a MEMS mirror and laser \cite{dtof} \cite{9105183} \cite{pittaluga2020} \cite{8369664}, sampling specific depths with Temporal Gating \cite{10.1117/12.2078169} and the use of sampling specific depths using Triangulating Light Curtains \cite{bartels2019Agile} \cite{wang2018programmable}.

Probablistic Approaches to Active Sensing have also been well studied, with work on using probabilistic occupancy grids for beliefs over states in the context of NBV \cite{isler2016information} \cite{kriegel2015efficient} \cite{8369664} \cite{Ancha_2020_ECCV} \cite{daudelin2017adaptable}. Probablistic Depth Estimation from images alone, with Depth Distributions over rays have also been researched \cite{liu2019neural} \cite{yang2019inferring} \cite{chang2018pyramid}. Lastly, we also looked at Gated Depth Imaging used with Probablistic Depth frameworks \cite{walz2020uncertainty} \cite{gruber2019gated2depth}.

Based on this prior work, we want to see if we can use this Probablistic Depth Estimation representation to solve the problem of NBV in the context of Active Sensing, using Triangulating Light Curtains.